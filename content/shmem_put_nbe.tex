\apisummary{
    The explicit non-blocking put routines provide a method for copying data
    from a contiguous local data object to a data object on a specified \ac{PE}.
}

\begin{apidefinition}

\begin{C11synopsis}
void shmem_put_nbe(TYPE *dest, const TYPE *source, size_t nelems, int pe, shmem_request_t *req);
\end{C11synopsis}
where \TYPE{} is one of the standard \ac{RMA} types specified by Table~\ref{stdrmatypes}.

\begin{Csynopsis}
void shmem_<TYPENAME>_put_nbe(TYPE *dest, const TYPE *source, size_t nelems, int pe, shmem_request_t *req);
\end{Csynopsis}
where \TYPE{} is one of the standard \ac{RMA} types and has a
corresponding \TYPENAME{} specified by Table~\ref{stdrmatypes}.

\begin{CsynopsisCol}
void shmem_put<SIZE>_nbe(void *dest, const void *source, size_t nelems, int pe, shmem_request_t *req);
\end{CsynopsisCol}
where \SIZE{} is one of \CONST{8, 16, 32, 64, 128}.

\begin{CsynopsisCol}
void shmem_putmem_nbe(void *dest, const void *source, size_t nelems, int pe, shmem_request_t *req);
\end{CsynopsisCol}

\begin{apiarguments}
    \apiargument{OUT}{dest}{Data object to be updated on the remote \ac{PE}. This
    data object must be remotely accessible.}
    \apiargument{IN}{source}{Data object containing the data to be copied.}
    \apiargument{IN}{nelems}{Number of elements in the \VAR{dest} and \VAR{source}
    arrays. \VAR{nelems} must be of type \VAR{size\_t} for \Cstd. When using
    \Fortran, it must be a constant, variable, or array element of default
    integer type.}
    \apiargument{IN}{pe}{\ac{PE} number of the remote \ac{PE}. \VAR{pe} must be
    of type integer. When using \Fortran, it must be a constant, variable,
    or array element of default integer type.}
    \apiargument{IN/OUT}{req}{Either NULL, or a valid request object}
\end{apiarguments}

\apidescription{
    The explicit non-blocking get routines provide a method for copying a
    contiguous symmetric data object from the local \ac{PE} to a contiguous data
    object on a remote \ac{PE}.
    The routines return after posting the operation. If the request
    parameter is initializet to NULL, a new request will be created. If
    the parameter is a valid request object, the operation will be
    merged into the request object.
    The delivery of data words into the data object on the destination
    \ac{PE} may occur in any order.
    Furthermore, two successive put routines may deliver data out of
    order. The operation is considered complete after a subsequent call
    to \FUNC{shmem\_wait\_request}.
 }

\apidesctable{
    The \dest{} and \source{} data objects must conform to certain typing
    constraints, which are as follows:}
    {Routine}{Data type of \VAR{dest} and \VAR{source}}{
    \apitablerow{shmem\_putmem\_nbe}{Any  data  type.  nelems is
                                     scaled in bytes.}
    \apitablerow{shmem\_put4\_nbe, shmem\_put32\_nbe}{Any noncharacter type
        that has a storage size equal to \CONST{32} bits.}
    \apitablerow{shmem\_put8\_nbe}{Any noncharacter type that
        has a storage size equal to \CONST{8} bits.}
    \apitablerow{shmem\_put64\_nbe}{Any noncharacter type that
        has a storage size equal to \CONST{64} bits.}
    \apitablerow{shmem\_put128\_nbe}{Any noncharacter type that has a
        storage size equal to \CONST{128} bits.}
}
\apireturnvalues{
    None.
}

\apinotes{ None }

\end{apidefinition}
