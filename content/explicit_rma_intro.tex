The explicit non-blocking \ac{RMA} routines described in this section are
one-sided communication mechanisms similar to the \ac{RMA} routines
described in Section~\ref{sec:rma}.
It describes the routines for: explicit non-blocking \ac{RMA},
the routines used to track the \ac{RMA} routines, and the
allocation, deallocation and merging of request objects.

All explicit non-blocking \ac{RMA} routines take a handle to a request
object as parameter. Request objects are of the type
\CTYPE{shmem\_request\_t}. The request object is used to track the
status of \ac{RMA} operations. A request object can be used to track one
or more operations. If a request object is used for multiple
communication operations, the resources assigned to the request object
can be shared.

Non-blocking \ac{RMA} routines are safe to use in multi threaded
environments. Merged requests can be used to separate communication
streams on a per thread basis, or can be shared among threads.
