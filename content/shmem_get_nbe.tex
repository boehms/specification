\apisummary{
    The explicit non-blocking get routines provide a method for copying
    data from a contiguous remote data object on the specified \ac{PE}
    to the local data object.
}

\begin{apidefinition}

\begin{C11synopsis}
void shmem_get_nbe(TYPE *dest, const TYPE *source, size_t nelems, int pe, shmem_request_t *req);
\end{C11synopsis}
where \TYPE{} is one of the standard \ac{RMA} types specified by Table \ref{stdrmatypes}.

\begin{Csynopsis}
void shmem_<TYPENAME>_get_nbe(TYPE *dest, const TYPE *source, size_t nelems, int pe, shmem_request_t *req);
\end{Csynopsis}
where \TYPE{} is one of the standard \ac{RMA} types and has a corresponding \TYPENAME{} specified by Table \ref{stdrmatypes}.

\begin{CsynopsisCol}
void shmem_get<SIZE>_nbe(void *dest, const void *source, size_t  nelems, int pe, shmem_request_t *req);
\end{CsynopsisCol}
where \SIZE{} is one of \CONST{8, 16, 32, 64, 128}.

\begin{CsynopsisCol}
void shmem_getmem_nbe(void *dest, const void *source, size_t nelems, int pe, shmem_request_t *req);
\end{CsynopsisCol}

\begin{apiarguments}
    \apiargument{OUT}{dest}{Local data object to be updated.}
    \apiargument{IN}{source}{Data object on the \ac{PE} identified by \VAR{pe}
        that contains the data to be copied.  This data object must be remotely
        accessible.}
    \apiargument{IN}{nelems}{Number of elements in the \dest{} and \source{}
        arrays. \VAR{nelems} must be of type \VAR{size\_t}.}
    \apiargument{IN}{pe}{Identifier of the remote \ac{PE}. \VAR{pe} must
        be of type integer.}
    \apiargument{IN/OUT}{req}{Either NULL, or a valid request object}
\end{apiarguments}

\apidescription{
    The explicit non-blocking get routines provide a method for copying
    a contiguous symmetric data object from a different \ac{PE} to a
    contiguous data object on the local \ac{PE}. The routines return
    after posting the operation.  The operation is considered complete
    after a subsequent call to \FUNC{shmem\_wait\_request}. After the
    completion of \FUNC{shmem\_wait\_request}, the data has been delivered
    to the \dest{} array on the local \ac{PE}.
}

\apidesctable{
    The  \dest{} and \source{} data objects must conform to typing constraints,
    which are as follows:
}{Routine}{Data type of \VAR{dest} and \VAR{source}}{
    \apitablerow{shmem\_getmem\_nbe}{Any  data  type.  nelems is scaled in bytes.}
    \apitablerow{shmem\_get8\_nbe}{Any noncharacter type that has a storage size equal to \CONST{8} bits.}
    \apitablerow{shmem\_get32\_nbe}{Any noncharacter type that has a storage size equal to \CONST{32} bits.}
    \apitablerow{shmem\_get64\_nbe}{Any noncharacter type that has a storage size equal to \CONST{64} bits.}
    \apitablerow{shmem\_get128\_nbe}{Any noncharacter type that has a storage size equal to \CONST{128} bits.}
}
\apireturnvalues{
    None.
}

\apinotes{
    See Section \ref{subsec:memory_model} for a definition of the term
    remotely accessible.
}

\end{apidefinition}
