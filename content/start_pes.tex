\apisummary{ 
    Called at the beginning of an \openshmem program to initialize the execution
    environment. This routine is deprecated and is provided for backwards
    compatibility. Implementations must include it, and the routine should
    function properly and may notify the user about deprecation of its use.
}

\begin{apidefinition}

\begin{DeprecateBlock}
\begin{Csynopsis}
void start_pes(int npes);
\end{Csynopsis}
\end{DeprecateBlock}

\begin{Fsynopsis}
CALL START_PES(npes)
\end{Fsynopsis}

\begin{apiarguments}
       \apiargument{npes}{Unused}{ Should be set to \CONST{0}.}
\end{apiarguments}

\apidescription{   
     The \FUNC{start\_pes} routine initializes the \openshmem execution
     environment.  An \openshmem program must call either \FUNC{start\_pes} or
     \FUNC{shmem\_init} before calling any other \openshmem routine.  Unlike
     \FUNC{shmem\_init}, \FUNC{start\_pes} does not require a call to
     \FUNC{shmem\_finalize}.  Instead, the \openshmem library is implicitly
     finalized when the program exits.  Implicit finalization is collective and
     includes a global synchronization to ensure that all pending communication
     is completed before resources are released.
}

\apireturnvalues{
    None.
}

\apinotes{
    If any other \openshmem call occurs before \FUNC{start\_pes}, the
    behavior is undefined.  Although it is recommended to set \VAR{npes} to
    \CONST{0} for \FUNC{start\_pes}, this is not mandated.  The value is ignored.
    Calling \FUNC{start\_pes} more than once has no subsequent
    effect.

    As of \openshmem Specification 1.2 the use of \FUNC{start\_pes} has
    been deprecated. Although \openshmem libraries are required to support the
    call, program users are encouraged to use \FUNC{shmem\_init} instead.
}


\begin{apiexamples}

\apicexample
    { This is a simple program that calls \FUNC{start\_pes}:}
    {./example_code/shmem_startpes_example.f90}
    {} 

\end{apiexamples}

\end{apidefinition}
